\section{Introduction}
% The introduction should include a summary of the objective of the project and quickly
% describe your progress and what you were able to achieve. The introduction should include a brief description
% of the relevant techniques and a high-level overview of your implementation. The introduction
% should also include a brief qualitative and quantitative overview of the results and point out a few key
% insights from the project.
For this project, we implemented AES, RSA, SHA-512, and a Unix password cracking algorithm on an FPGA.
RSA proved very difficult to implement, and ultimately was far slower on an FPGA than an optimized software
version running on the ZedBoard's CPU -- about 30 times slower for encryption.
However, AES, SHA-512, and the Unix password cracker were very amenable
to FPGA implementation. While neither was able to beat a server-grade Intel CPU (ecelinux) in terms of raw performance,
all 3 achieved speedup with respect to the ZedBoard's CPU. AES had a speedup of about 1.5 times (depending on the variant of AES),
while the password hasher had a speedup of 5.2 times.
However, for AES and the password cracker, there are significant power advantages from using an FPGA.
While the password cracker running on ecelinux processors required 0.542 $\frac{\text{J}}{\text{Hash}}$, the optimized
FPGA version required only 0.003 $\frac{\text{J}}{\text{Hash}}$.

For each algorithm, we implemented an optimized software version, and then an FPGA version.
We benchmarked each algorithm on the server grade Intel CPUs (ecelinux), the embedded ZedBoard CPUs,
and then on the ZedBoard FPGA. Each algorithm is explained in its own section below.
