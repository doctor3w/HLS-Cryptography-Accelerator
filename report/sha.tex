\section{SHA}
The Secure Hash Algorithm (SHA) is a hash function used in a wide range of applications, from checking
file integrity to securely storing passwords. There are a number of versions of SHA which each generate
different sized hashes. The original SHA algorithm, SHA-1, produced 160-bit hashes, while SHA-512,
currently used by Linux to hash passwords, produces 512-bit hashes. For this project we decided to
implement SHA-512 since our end goal was to create a Unix password cracker.
 
\subsection{Techniques}
% This section should provide a detailed description of the applications, algorithms, or
% hardware architectures realized in this project. Think critically about the important items to mention
% in order for the reader to understand how your design works without having to look into any code.
% For example, what are the inputs and outputs of the application (or architecture), what are the major
% steps (or modules), and what does each step (or module) achieve? It would be useful to include
% small examples, block diagrams, mathematical formulas, and other visualizations to help explain your
% techniques. Do not include detailed information about your source code as your report should be at a
% high level.
\subsection{SHA-512}
\subsection{Unix Password Hashing}

\subsection{Implementation}
% This section describe how you implemented your designs. For example, what
% programming languages did you use? Did you take advantage of any third-party libraries? Is your
% implementation purely software, purely hardware, or a mix of both? Which software and/or hardware
% blocks are included in your design, and what hardware device (if any) did you target? In most cases, it
% would be helpful to include block diagrams of your implementation illustrating the flow of data through
% your design, the interconnection between different blocks, and whether each block is implemented
% in software or hardware. As in the previous section, providing meaningful visualizations would help
% the reader better appreciate your work. Please also include one or two interesting aspects of your
% implementation, especially any specific implementation strategies necessary for creating a functionally
% correct design with good performance.
\subsection{SHA-512}
\subsection{Unix Password Hashing}

\subsection{Evaluation}
% Students should describe the experimental setup used to evaluate their design. Students
% should describe the data inputs used to evaluate their design and provide an analysis of the achieved
% results. The results should be clearly summarized in terms of tables, text, and/or plots. Please provide
% qualitative and quantitative analysis of the results and discuss insights from these results. Results may
% include (but are not limited to) the execution time of an algorithm, hardware resource usage, achievable
% throughput, and error rate. It would be interesting, for example, to discuss why one design is better
% than another, why one design achieves a higher metric than another, or how you trade-off one metric
% for another. Consider going into detail for one particular instance of your experiment and analyze how
% it achieves the given results.

