\section{SHA512}
The Secure Hash Algorithm (SHA) is a crytographic hash function used in a wide range of applications, from checking
file integrity to securely storing passwords. Unlike AES or RSA, which are usually used for
encrypting messages, SHA hashes an arbitraty size message into a fixed-size irreversable hash.
The original SHA algorithm, SHA-1, produced 160-bit hashes, while SHA-512
produces 512-bit hashes. From these hashes it is nearly impossible to determine
what the original message was.

For this project we decided to implement SHA-512 on an FPGA.
We then use it to accelerate unix password hashing and implemement
a brute-force hardware accelerated Unix password cracker.

\subsection{Techniques}
% This section should provide a detailed description of the applications, algorithms, or
% hardware architectures realized in this project. Think critically about the important items to mention
% in order for the reader to understand how your design works without having to look into any code.
% For example, what are the inputs and outputs of the application (or architecture), what are the major
% steps (or modules), and what does each step (or module) achieve? It would be useful to include
% small examples, block diagrams, mathematical formulas, and other visualizations to help explain your
% techniques. Do not include detailed information about your source code as your report should be at a
% high level.
\subsubsection{SHA-512}
SHA-512 requires some basic preprocessing of the message before it is hashed. The first
step is appending a single 1-bit to the buffer containing the message immediately after the message itself.
In most practical cases the message is a integer number of bytes, so this step is just appending
a single $0x80$ byte to the end. The final preprocessing step is appending the
length of the message (measured in bits) to the end of the buffer.
Since the algorithm operates on 1024-bit blocks, the buffer storing the preprocessed message
must be an integer number of 1024-bit blocks. Thus, the last 128-bits of the last block
is reserved for storing the message length in big-endian format. Zero padding is
applied to the buffer (if needed) from after where the $0x80$ was appended until the last 16-bytes
of the final block is reached, from which the message length is then written.


The core of SHA-512 is the compression function, which takes an intermediate hash value and a single 1024-bit
block of data and returns another hash value which is essentially the intermediate hash value encrypted
with the message as the key. After the preprocessing step above,
each block is then processed sequentually using the
intermediate hash of the previous block.
For the first block, the hash function is called with a fixed initial value. Then for each subsequent
block, the intermediate hash passed into the compression function is the result of the previous run of the
compresssion computation. That is, any given intermediate hash value directly depends on the preceeding
intermediate hash. The outputted hash is just the hash computed for the last block.

The algorithm for the compression function is given in \ref{SHA512}. Let $H^i$ denote the current
intermediate hash value, and $H^i_n$ be the $n^{th}$ 64-bit word in that hash. Let $M^i$ denote the
$i^{th}$ 1024-bit chunk of the message, and let $M^i_j$ denote the $j^{th}$ 64-bit word in that chunk (treating it as big-endian).
For our purposes, the exact definitions of $\Sigma_0, \Sigma_1, \sigma_0, \sigma_1, Maj(a,b,c),
Ch(e,f,g)$ are not relevant. It is sufficient to understand that they are combinations of various bitwise
operations including shifting and XOR-ing.

\begin{algorithm}[!htb]
  \label{SHA512}
  \caption{SHA-512 Procedure for hashing a block. Adapted from \url{http://www.iwar.org.uk/comsec/resources/cipher/sha256-384-512.pdf}}
  \begin{algorithmic}
    \Procedure{hashBlock}{}
    \State $a \leftarrow H^{i-1}_1$
    \Comment Assign $a..h$ to the corresponding 64-bit chunks of the intermediate hash
    \State $\vdots$
    \State $h \leftarrow H^{i-1}_8$
    \State $W_j \leftarrow M_j^i$ for $j=0,1,...,15$
    \Comment The first 16 64-bit words of $W$ are from the message block
    \For{$j = 16$ to $79$}
    \Comment The remaining 64 are computed recursively
      \State $W_i \leftarrow \sigma_1(W_{i-2}) + W_{i-7} + \sigma_0(W_{j-15}) + W_{j-16}$
    \EndFor
    \For{$j = 1$ to $79$}
      \Comment Apply the compression function to compute the output hash
      \State $T_1 \leftarrow h + \Sigma_1(e) + Ch(e,f,g) + K_j + W_j$
      \State $T_2 \leftarrow \Sigma_0(a) + Maj(a,b,c)$
      \State $h \leftarrow g$
      \State $g \leftarrow f$
      \State $f \leftarrow e$
      \State $e \leftarrow d + T_1$
      \State $d \leftarrow c$
      \State $c \leftarrow b$
      \State $b \leftarrow a$
      \State $a \leftarrow T_1 + T_2$
    \EndFor
    \State $H_1^i \leftarrow a + H^{i-1}_1$
    \Comment Assemble the new hash using the computed values
    \State $\vdots$
    \State $H_8^i \leftarrow a + H^{i-1}_8$
    \EndProcedure
  \end{algorithmic}
\end{algorithm}

\subsubsection{Unix Password Hashing}
To test our hardware SHA-512 we were implementing, the group decided to create a
Unix password cracker. Most unix systems store the paswords hashes in the
``/etc/shadow'' file. Entires in this file have the format: ``[user]:\$[hash\_algorithm]\$salt\$hash''.
Modern systems usually use SHA-512 for the underlying hash\_algorithm which is represented
with a ``\$6\$''. The algorithm, for generating for converting a Unix password
into a hash is shown in \ref{SHA512-unix}. This algorithm is normally
accessed through the $crypt()$ function in $<unistd.h>$.

\begin{algorithm}[!htb]
  \label{SHA512-unix}
  \caption{Unix SHA-512 $crypt()$ implementation. Adapted from \url{https://akkadia.org/drepper/SHA-crypt.txt}}
  \begin{algorithmic}
    \Procedure{crypt\_sha512()}{}
    \State B = sha512(password + salt + password)
    \State A = sha512(password + salt + B[:pwlen])
    \State DP = sha512(password * pwlen)
    \State DS = sha512(salt * (16 + A[0]))
    \For{$j = 1$ to $4999$}
      \State buf = []
      \State buf.append(DP[:pwlen] if (i \% 2) else A)
      \State buf.append(DS[:saltlen] if (i \% 3) else ‘’)
      \State buf.append(DP[:pwlen] if (i \% 7) else ‘’)
      \State buf.append(DP[:pwlen] if (i \% 2 == 0) else A)
      \State A = sha512(buf)
    \EndFor
    \State return base64Encode(A)
    \EndProcedure
  \end{algorithmic}
\end{algorithm}

The key takeaway from the SHA-512 Unix password hashing algorithm is
that it requires computing $5000$ SHA-512 hashes of variable length messages,
where each hash is dependent on the previous one. It is extremely slow
and expensive to compute. When testing through the $crypt()$ function
on the zedboard's ARM core, it was only able to hash roughly 10 passwords
per second.

\subsection{Implementation}
% This section describe how you implemented your designs. For example, what
% programming languages did you use? Did you take advantage of any third-party libraries? Is your
% implementation purely software, purely hardware, or a mix of both? Which software and/or hardware
% blocks are included in your design, and what hardware device (if any) did you target? In most cases, it
% would be helpful to include block diagrams of your implementation illustrating the flow of data through
% your design, the interconnection between different blocks, and whether each block is implemented
% in software or hardware. As in the previous section, providing meaningful visualizations would help
% the reader better appreciate your work. Please also include one or two interesting aspects of your
% implementation, especially any specific implementation strategies necessary for creating a functionally
% correct design with good performance.
\subsubsection{SHA-512}
Before we created the password cracker, we implemented the SHA-512 core on its own, first to ensure
correctness, and then as a way to optimize it in isolation. We implemented SHA-512 in software first,
using the algorithm presented in the``Descriptions of SHA-256, SHA-384, and SHA-512'' paper that is linked
in the description of \ref{SHA512}. We then modified it as needed to make it
sythesizable. Ultilmately, we ended up with \emph{SHA512Hasher}, a C++ class that exposed the following methods:
\begin{verbatim}
  void reset(); // Reset the state of the hasher
  void update(const uint8_t *msgp, uint8_t len); // Append msg to the hasher
  void byte_digest(uint8_t buf[64]); // Output the SHA-512 hash
\end{verbatim}

Internally, the contained a length 128 byte buffer (to store the current block)
as well the current intermediate hash,
which was initilized to the inital SHA-512 intermediate value. This buffer would be filled as
\emph{update()} was repeadedly called on it. Whenever the buffer filled up, the class would
automatically (in the \emph{update()} call) run the SHA-512 algorithm for the current block
and compute the next intermediate hash value based on this current block and the current intermediate hash.
This was implemented as a private method called \emph{hashBlock()}.
When \emph{byte\_digest()} was called, the hasher would do the final preprocessing steps for the last block,
and compute the final SHA-512 hash. A \emph{reset()} method was also provided to allow the same hasher
instance to be reused to compute further hashes.


This implementaiton works extremely well for an HLS FPGA implementation, because
the entire message does not have to be stored. Only a temporary 128 byte buffer
to store the current block, and a 64-byte buffer to store the intermediate hash for the previous block is needed.
Furthermore, by doing this, it allows the size of the message to be variable and be unknown at compile time.
This makes the implementation extremely general and thus can easily be adopted to
other projects. This generalness was infact needed later to support variable length passwords with the Unix password
hasher that was implemented later.

Since the slow part of the algorithm was was performing the 80 SHA-512 rounds (\ref{SHA512}),
optimizing this \emph{hashBlock()} method was the first priority. Initally, its
latency was $534$ cycles. However, a number of optimizations was applied that
brought the latency down to $88$ cycles.

One of the most important optimizations
involved recognizing that the $i^{th}$ value of $W$ (\ref{SHA512}) only relies on the 15 preceeding $W$ values.
Since each block of $W$ is accessed sequentially, we can use a 16-block block array as a shift register
to store the values we need, instead of the entire 80 block array. Since the first 16 values in $W$ are
just the current message chunk with no additional processing, we can simultaneously do the first 16
rounds of the compression function and also fill $W$ in parallel, and then do the remaining $64$ rounds
computing one new $W$ value each time. We fully partitioned this $W$ array, and removed the need
to read/write from a BRAM each cycle. Furthermore, by implementing $W$ with a shift register,
each loop iteration would be reading and writing to the same registers every time,
as opposed to a different BRAM address.

The final set of optimizations invovled partitioning the 128 byte block buffer
cyclically by a factor of 8. This allowed the first length 16 loop discussed
in the previous paragraph, to initlize each $W_i$ in a single cycle. Next, the
intermediate value hash array was fully partitioned so that all $a \ldots h$ could
be initlized at the beggening of \emph{hashBlock()} as well as updated
at the end ind a single cycle. Finally both the 16 and 64 iteration loop were
pipelined, and due to the previous optimizations, allowed them to be II1.


At this point \emph{hashBlock()} had a latency of 88 cycles, which was quite good
considering the SHA-512 algorithm itself consists of 80 iterations,
where each iteration was dependent on the previous one. Thus, after these optimizations,
there was practically no further parallelism that could be exploited in \emph{hashBlock()}.


\subsubsection{Unix Password Hashing}


\subsection{Evaluation}
% Students should describe the experimental setup used to evaluate their design. Students
% should describe the data inputs used to evaluate their design and provide an analysis of the achieved
% results. The results should be clearly summarized in terms of tables, text, and/or plots. Please provide
% qualitative and quantitative analysis of the results and discuss insights from these results. Results may
% include (but are not limited to) the execution time of an algorithm, hardware resource usage, achievable
% throughput, and error rate. It would be interesting, for example, to discuss why one design is better
% than another, why one design achieves a higher metric than another, or how you trade-off one metric
% for another. Consider going into detail for one particular instance of your experiment and analyze how
% it achieves the given results.
\begin{table}[h]
\begin{tabular}{@{}llllll@{}}
\toprule
Version                   & Latency (cycles) & BRAM Usage & DSP Usage & FF Usage & LUT Usage \\ \midrule
baseline                  & 534              & 2\%         & 0\%        & 0\%       & 5\%    \\
baseline-pipeline-1       & 299              & 2\%         & 0\%        & 1\%       & 6\%    \\
baseline-pipeline-2       & 235              & $<$1\%      & 0\%        & 9\%       & 20\%   \\
shift-register            & 2373             & 2\%         & 0\%        & 1\%       & 9\%    \\
shift-register-pipeline-1 & 91               & $<$1\%      & 0\%        & 3\%       & 12\%   \\
final-opt                 & 88               & $<$1\%      & 0\%        & 4\%       & 7\%   \\ \bottomrule
\end{tabular}
\caption{Usage and cycle counts for different version of our SHA-512 implementation. The baseline
version are nealy indentical to the software implementations, but they make use of optimization
directives. The shift-register versions use the optimization described in the Techniques section to
reduce the size of W and to remove the extraneous loops, in addition to making use of optimization
directives}
\label{table:shaversions}
\end{table}
