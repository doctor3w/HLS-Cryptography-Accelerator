\section{SHA512}
The Secure Hash Algorithm (SHA) is a cryptographic hash function used in a wide range of applications, from checking
file integrity to securely storing passwords.
At a high level, SHA is a function from an arbitrary size message into a fixed-size hash.
Furthermore, the function is very difficult to invert: given a hash, determining the original 
message is nearly impossible.
The original SHA algorithm, SHA-1, produced 160-bit hashes, while SHA-512
produces 512-bit hashes.

We implemented SHA-512 on an FPGA, used it to accelerate Unix password hashing, and then implemented
a simple brute-force hardware accelerated Unix password cracker.

\subsection{Techniques}
% This section should provide a detailed description of the applications, algorithms, or
% hardware architectures realized in this project. Think critically about the important items to mention
% in order for the reader to understand how your design works without having to look into any code.
% For example, what are the inputs and outputs of the application (or architecture), what are the major
% steps (or modules), and what does each step (or module) achieve? It would be useful to include
% small examples, block diagrams, mathematical formulas, and other visualizations to help explain your
% techniques. Do not include detailed information about your source code as your report should be at a
% high level.
\subsubsection{SHA-512}
The core of SHA-512 is the compression function, which takes an intermediate hash value and a single 1024-bit
block of data and returns another hash value.
For the first block, the hash function is called with a fixed initial value. Then for each subsequent
block, the intermediate hash passed into the compression function is the result of the previous run of the
compression computation. That is, any given intermediate hash value directly depends on the preceding
intermediate hash.

At the end of the message, the algorithm appends a single "0x80" byte. The last 128 bits of the final block
are set to the length of the message in bits, as a 128 bit big-endian number. The space between the "0x80" and
the start of the length is zero-padded.
The final hash is just the hash computed for the last block.

The algorithm for the compression function is given in \ref{SHA512}. Let $H^i$ denote the current
intermediate hash value, and $H^i_n$ be the $n^{\text{th}}$ 64-bit word in that hash. Let $M^i$ denote the
$i^{\text{th}}$ 1024-bit chunk of the message, and let $M^i_j$ denote the $j^{\text{th}}$ 64-bit word in that chunk (treating it as big-endian).
For our purposes, the exact definitions of $\Sigma_0, \Sigma_1, \sigma_0, \sigma_1, Maj(a,b,c),
Ch(e,f,g)$ are not relevant. It is sufficient to understand that they are various bit twiddling functions involving 
many shift and XOR operations.

\begin{algorithm}[!htb]
  \label{SHA512}
  \caption{SHA-512 Procedure for hashing a block. Adapted from \url{http://www.iwar.org.uk/comsec/resources/cipher/sha256-384-512.pdf}}
  \begin{algorithmic}
    \Procedure{hashBlock}{}
    \State $a \leftarrow H^{i-1}_1$
    \Comment Assign $a..h$ to the corresponding 64-bit chunks of the intermediate hash
    \State $\vdots$
    \State $h \leftarrow H^{i-1}_8$
    \State $W_j \leftarrow M_j^i$ for $j=0,1,...,15$
    \Comment The first 16 64-bit words of $W$ are from the message block
    \For{$j = 16$ to $79$}
    \Comment The remaining 64 are computed recursively
      \State $W_i \leftarrow \sigma_1(W_{i-2}) + W_{i-7} + \sigma_0(W_{j-15}) + W_{j-16}$
    \EndFor
    \For{$j = 1$ to $79$}
      \Comment Apply the compression function to compute the output hash
      \State $T_1 \leftarrow h + \Sigma_1(e) + Ch(e,f,g) + K_j + W_j$
      \State $T_2 \leftarrow \Sigma_0(a) + Maj(a,b,c)$
      \State $h \leftarrow g$
      \State $g \leftarrow f$
      \State $f \leftarrow e$
      \State $e \leftarrow d + T_1$
      \State $d \leftarrow c$
      \State $c \leftarrow b$
      \State $b \leftarrow a$
      \State $a \leftarrow T_1 + T_2$
    \EndFor
    \State $H_1^i \leftarrow a + H^{i-1}_1$
    \Comment Assemble the new hash using the computed values
    \State $\vdots$
    \State $H_8^i \leftarrow a + H^{i-1}_8$
    \EndProcedure
  \end{algorithmic}
\end{algorithm}

\subsubsection{Unix Password Hashing}
Most unix systems store the passwords hashes in the "/etc/shadow" file. Entries in this file have the format: "[user]:\$[hash_algorithm]\$salt\$hash".
Modern systems usually use SHA-512 for the underlying "hash_algorithm" which is represented
with a "6". The algorithm for hashing a Unix password is shown in \ref{SHA512-unix}. This algorithm is normally
accessed through the "crypt()" function in "<unistd.h>".

\begin{algorithm}[!htb]
  \label{SHA512-unix}
  \caption{Unix SHA-512 $crypt()$ implementation. Adapted from \url{https://akkadia.org/drepper/SHA-crypt.txt}}
  \begin{algorithmic}
    \Procedure{crypt\_sha512()}{}
    \State B = sha512(password + salt + password)
    \State A = sha512(password + salt + B[:pwlen])
    \State DP = sha512(password * pwlen)
    \State DS = sha512(salt * (16 + A[0]))
    \For{$j = 1$ to $4999$}
      \State buf = []
      \State buf.append(DP[:pwlen] if (i \% 2) else A)
      \State buf.append(DS[:saltlen] if (i \% 3) else ‘’)
      \State buf.append(DP[:pwlen] if (i \% 7) else ‘’)
      \State buf.append(DP[:pwlen] if (i \% 2 == 0) else A)
      \State A = sha512(buf)
    \EndFor
    \State return base64Encode(A)
    \EndProcedure
  \end{algorithmic}
\end{algorithm}

Note that the SHA-512 Unix password hashing algorithm requires computing 5000
SHA-512 hashes of variable length messages, where each hash is dependent on the previous one.
It is extremely slow operation; the ZedBoard's ARM core can only hash about 10 passwords per second.

\subsection{Implementation}
% This section describe how you implemented your designs. For example, what
% programming languages did you use? Did you take advantage of any third-party libraries? Is your
% implementation purely software, purely hardware, or a mix of both? Which software and/or hardware
% blocks are included in your design, and what hardware device (if any) did you target? In most cases, it
% would be helpful to include block diagrams of your implementation illustrating the flow of data through
% your design, the interconnection between different blocks, and whether each block is implemented
% in software or hardware. As in the previous section, providing meaningful visualizations would help
% the reader better appreciate your work. Please also include one or two interesting aspects of your
% implementation, especially any specific implementation strategies necessary for creating a functionally
% correct design with good performance.
\subsubsection{SHA-512}
We implemented SHA-512 in software first, using the algorithm presented in the ``Descriptions of SHA-256, SHA-384, and SHA-512'' (see algorithm \ref{SHA512})
We then modified it as needed to make it synthesizable. Ultimately, we ended up with \emph{SHA512Hasher}, a C++ class that exposed the following methods:
\begin{verbatim}
  void reset(); // Reset the state of the hasher
  void update(const uint8_t *msgp, uint8_t len); // Append msg to the hasher
  void byte_digest(uint8_t buf[64]); // Output the SHA-512 hash
\end{verbatim}

Internally, the class contains a length 128 byte buffer (to store the current block)
as well the current intermediate hash,
which is initialized to the initial SHA-512 intermediate value. This buffer is filled as
"update()" is repeatedly called on it. Whenever the buffer fills up, the class
automatically (in the "update()" call) runs the SHA-512 algorithm for the buffered block
and computes the next intermediate hash value based on this current block and the stored intermediate hash hash of previous block.
This was implemented as a private method called "hashBlock()".
When "byte_digest()" is called, the hasher does the final steps for the last block,
and computes the final SHA-512 hash. A "reset()" method was also provided to allow the same hasher
instance to be reused to compute further hashes.

This implementation works extremely well for an HLS FPGA implementation, because
the entire message does not have to be stored. Only a temporary 128 byte buffer
to store the current block, and a 64-byte buffer to store the intermediate hash for the previous block is needed.
Furthermore, by doing this, it allows the size of the message to be variable and be unknown at compile time.

Since the slow part of the algorithm was was performing the 80 SHA-512 rounds,
optimizing this "hashBlock()" method was the first priority. Initially, its
latency was 534 cycles. However, the optimized version was only 88 cycles.

One of the most important optimizations
involved recognizing that the $i^{\text{th}}$ value of $W$ (algorithm \ref{SHA512}) only relies on the 15 preceding $W$ values.
Since each block of $W$ is accessed sequentially, we can use a 16-block block array as a shift register
to store the values we need, instead of the entire 80 block array. Since the first 16 values in $W$ are
just the current message chunk with no additional processing, we can simultaneously do the first 16
rounds of the compression function and also fill $W$ in parallel, and then do the remaining $64$ rounds
computing one new $W$ value each time. We fully partitioned this $W$ array, and removed the need
to read/write from a BRAM each cycle. Furthermore, by implementing $W$ with a shift register,
each loop iteration would be reading and writing to the same registers every time,
as opposed to a different BRAM address.

The final set of optimizations involved partitioning the 128 byte block buffer
cyclically by a factor of 8. This allowed the first length 16 loop discussed
in the previous paragraph to initialize each $W_i$ in a single cycle. Next, the
intermediate value hash array was partially partitioned so that all $a \ldots h$ could
be initialized at the beginning of "hashBlock()" as well as updated
at the end in a single cycle. Finally both the 16 and 64 iteration loop were
pipelined, and due to the previous optimizations, allowed them to have an II of 1.

At this point "hashBlock()" had a latency of 88 cycles, which was quite good
considering the SHA-512 algorithm itself consists of 80 iterations,
where each iteration is dependent on the previous one. Thus, after these optimizations,
there was practically no further parallelism that could be exploited.

\subsubsection{Unix Password Hashing}
The entirety of the SHA-512 portion of the "crypt()" function was implemented
on the FPGA in "unix_cracker.cpp". The implementation was relatively straight
forward, and closely resembles the pseudo code in algorithm \ref{SHA512-unix}.
Due to the lack of remaining resources on the Zynq's FPGA, only minor
optimizations were applied to this portion of the implementation.
Additionally, the vast majority of the compute time was spent in the
"SHA512Hasher" methods, so resources were conserved and instead used to enable the
"SHA512Hasher" optimizations explained previously.

A basic host program was written, "zedboard/host.cpp", which sends a
salt and password (null-terminator separated) over the Xillybus. The FPGA then
sends the resulting hash back. The host program is responsible for
generating password guesses, sending to the FPGA, then comparing the returned
result against the known hash extracted from the "etc/shadow" file.
If these two hashes match, then the password tested was correct.

\subsection{Evaluation}
% Students should describe the experimental setup used to evaluate their design. Students
% should describe the data inputs used to evaluate their design and provide an analysis of the achieved
% results. The results should be clearly summarized in terms of tables, text, and/or plots. Please provide
% qualitative and quantitative analysis of the results and discuss insights from these results. Results may
% include (but are not limited to) the execution time of an algorithm, hardware resource usage, achievable
% throughput, and error rate. It would be interesting, for example, to discuss why one design is better
% than another, why one design achieves a higher metric than another, or how you trade-off one metric
% for another. Consider going into detail for one particular instance of your experiment and analyze how
% it achieves the given results.
\begin{table}[h]
\centering
\begin{tabular}{@{}llllll@{}}
\toprule
Version                   & Latency (cycles) & BRAM Usage & DSP Usage & FF Usage & LUT Usage \\ \midrule
baseline                  & 534              & 2\%         & 0\%        & 0\%       & 5\%    \\
baseline-pipeline-1       & 299              & 2\%         & 0\%        & 1\%       & 6\%    \\
baseline-pipeline-2       & 235              & $<$1\%      & 0\%        & 9\%       & 20\%   \\
shift-register            & 2373             & 2\%         & 0\%        & 1\%       & 9\%    \\
shift-register-pipeline-1 & 91               & $<$1\%      & 0\%        & 3\%       & 12\%   \\
final-opt                 & 88               & $<$1\%      & 0\%        & 4\%       & 7\%   \\ \bottomrule
\end{tabular}
\caption{Usage and cycle counts for different version of our SHA-512 implementation. The baseline
versions are nearly identical to the software implementations, but they make use of optimization
directives. The shift-register versions use the optimization described in the Techniques section to
reduce the size of $W$ and to remove the extraneous loops, in addition to making use of optimization
directives.}
\label{table:shausage}
\end{table}

\begin{table}[h]
\centering
\begin{tabular}{@{}lllllll@{}}
\toprule
  Version   & Average Time (s) & Hash/sec & Speedup & Price/Performance & Power/Performance (J/Hash) \\ \midrule
  ecelinux  & 0.00346          & 229      & 24      & 7.64  & 0.0655                        \\
  fpga-opt2 & 0.0200           & 50.1     & 3.5     & 5.99  & 0.00200                       \\
  fpga-opt1 & 0.0300           & 33.3     & 5.2     & 9.90  & 0.00300                       \\
  zedboard  & 0.104            & 9.62     & 1       & 31.2  & 0.00515                       \\ \bottomrule
\end{tabular}
\caption {Results for the Unix Password Hasher. The ecelinux and ZedBoard versions were identical software
  implementations, compiled with -O3. The fpga-opt1 and fpga-opt2 were differently optimized FPGA
  implementations of the password hashing algorithm. The speedups are given relative to the ZedBoard ARM
  CPU. The Price/Performance column is the ratio of the cost of the CPU to the number of hashes they computed
  per second. Prices are from manufacturer's recommended pricing. The Intel CPU cost \$1750 while the
  entire Zedboard (ARM CPU and FPGA) cost \$300. The Power/Performance column is the ratio of the average
  power consumption to the number of hashes they computed per second. The average power consumption
  information was taken from the manufacturers' websites. The Intel CPU has a TDP of 120W, and Xilinx 
  claims the ARM CPU and FPGA both have sub-watt TDPs of approximately 100mW. We divided the TDP by the
  number of cores for the Intel and ARM CPUs, and are assuming a linear speedup with multiple cores.}
\label{table:hashresults}
\end{table}

In order to evaluate our password hasher, we tested it on three different platforms: the Intel CPU on an
ecelinux server, the ARM CPU on a Zedboard, and the Zedboard FPGA. For the evaluation we used a host program
that simulated using the password hasher for a dictionary attack. The host program reads passwords from
a text file of commonly used passwords, and then sends the password as plaintext to the FPGA, which
then sends back the hash of the password as a base64 number represented with ASCII characters. For
evaluating the performance of the program on the ecelinux and ZedBoard CPUs, we instead used the 
"crypt()" from "<unistd.h>" function provided by the operating system to generate the hashes. On
ecelinux, we found that our test program was able to hash ~230 passwords per second. While this may seem
low, remember that each call to "crypt()" is computationally expensive since it performs 5000 rounds of
SHA-512. On the ZedBoard's ARM CPU, we could only perform ~10 hashes per second. The difference in speed
is surprising, given that the ZedBoard CPU only operates at half the clock speed of the ecelinux CPU, and
that "crypt()" is not parallelizable, so it doesn't benefit from the larger number of cores available on
ecelinux. Most likely it is due to differences in the implementation, the x86 version might use special
instructions that are not in the ARM ISA to speed up the computation. The optimized version of the FPGA,
described in the previous section, was able to achieve 50 hashes per second, which is over a 5x
improvement over the ZedBoard CPU. Even though the Intel CPU beats the FPGA in terms of raw
power, it loses when it comes to the ratio of price/performance and power/performance. The most optimized
version of the FPGA hasher achieves a price-to-performance ratio of 5.99, while the Intel CPU's ratio is
7.64 (lower is better). The most striking difference is in the power-to-performance scores. The Intel CPU
has a ratio of 0.524, while the FPGA achieves .00200, \emph{nearly two orders of magnitude lower} than the
Intel CPU. This our makes the FPGA-based implementation the clear choice for embedded applications, where
power consumption is a major concern.
