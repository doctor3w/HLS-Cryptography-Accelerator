\section{SHA512}
The Secure Hash Algorithm (SHA) is a crytographic hash function used in a wide range of applications, from checking
file integrity to securely storing passwords. Unlike AES or RSA, which are usually used for
encrypting messages, SHA hashes an arbitraty size message into a fixed-size irreversable hash.
The original SHA algorithm, SHA-1, produced 160-bit hashes, while SHA-512
produces 512-bit hashes. From these hashes it is nearly impossible to determine
what the original message was.

For this project we decided to implement SHA-512 on an FPGA.
We then use it to accelerate unix password hashing and implemement
a brute-force hardware accelerated Unix password cracker.

\subsection{Techniques}
% This section should provide a detailed description of the applications, algorithms, or
% hardware architectures realized in this project. Think critically about the important items to mention
% in order for the reader to understand how your design works without having to look into any code.
% For example, what are the inputs and outputs of the application (or architecture), what are the major
% steps (or modules), and what does each step (or module) achieve? It would be useful to include
% small examples, block diagrams, mathematical formulas, and other visualizations to help explain your
% techniques. Do not include detailed information about your source code as your report should be at a
% high level.
SHA-512 requires some basic preprocessing of the message before it is hashed. The first
step is appending a single 1-bit to the buffer containing the message immediately after the message itself.
In most practical cases the message is a integer number of bytes, so this step is just appending
a single $0x80$ byte to the end. The final preprocessing step is appending the
length of the message (measured in bits) to the end of the buffer.
Since the algorithm operates on 1024-bit blocks, the buffer storing the preprocessed message
must be an integer number of 1024-bit blocks. Thus, the last 128-bits of the last block
is reserved for storing the message length in big-endian format. Zero padding is
applied to the buffer (if needed) from after where the $0x80$ was appended until the last 16-bytes
of the final block is reached, from which the message length is then written.


The core of SHA-512 is the compression function, which takes an intermediate hash value and a single 1024-bit
block of data and returns another hash value which is essentially the intermediate hash value encrypted
with the message as the key. After the preprocessing step above,
each block is then processed sequentually using the
intermediate hash of the previous block.
For the first block, the hash function is called with a fixed initial value. Then for each subsequent
block, the intermediate hash passed into the compression function is the result of the previous run of the
compresssion computation. That is, any given intermediate hash value directly depends on the preceeding
intermediate hash.

The algorithm for the compression function is given in \ref{SHA512}. Let $H^i$ denote the current
intermediate hash value, and $H^i_n$ be the $n^{th}$ 64-bit word in that hash. Let $M^i$ denote the
$i^{th}$ 1024-bit chunk of the message, and let $M^i_j$ denote the $j^{th}$ 64-bit word in that chunk (treating it as big-endian).
For our purposes, the exact definitions of $\Sigma_0, \Sigma_1, \sigma_0, \sigma_1, Maj(a,b,c),
Ch(e,f,g)$ are not relevant. It is sufficient to understand that they are combinations of various bitwise
operations including shifting and XOR-ing.

\begin{algorithm}
  \label{SHA512}
  \caption{SHA-512 Procedure for hashing a block. Adapted from \url{http://www.iwar.org.uk/comsec/resources/cipher/sha256-384-512.pdf}}
  \begin{algorithmic}
    \Procedure{hashBlock}{}
    \State $a \leftarrow H^{i-1}_1$
    \Comment Assign $a..h$ to the corresponding 64-bit chunks of the intermediate hash
    \State $\vdots$
    \State $h \leftarrow H^{i-1}_8$
    \State $W_j \leftarrow M_j^i$ for $j=0,1,...,15$
    \Comment The first 16 64-bit words of $W$ are from the message block
    \For{$j = 16$ to $79$}
    \Comment The remaining 64 are computed recursively
      \State $W_i \leftarrow \sigma_1(W_{i-2}) + W_{i-7} + \sigma_0(W_{j-15}) + W_{j-16}$
    \EndFor
    \For{$j = 1$ to $79$}
      \Comment Apply the compression function to compute the output hash
      \State $T_1 \leftarrow h + \Sigma_1(e) + Ch(e,f,g) + K_j + W_j$
      \State $T_2 \leftarrow \Sigma_0(a) + Maj(a,b,c)$
      \State $h \leftarrow g$
      \State $g \leftarrow f$
      \State $f \leftarrow e$
      \State $e \leftarrow d + T_1$
      \State $d \leftarrow c$
      \State $c \leftarrow b$
      \State $b \leftarrow a$
      \State $a \leftarrow T_1 + T_2$
    \EndFor
    \State $H_1^i \leftarrow a + H^{i-1}_1$
    \Comment Assemble the new hash using the computed values
    \State $\vdots$
    \State $H_8^i \leftarrow a + H^{i-1}_8$
    \EndProcedure
  \end{algorithmic}
\end{algorithm}

\subsection{Implementation}
% This section describe how you implemented your designs. For example, what
% programming languages did you use? Did you take advantage of any third-party libraries? Is your
% implementation purely software, purely hardware, or a mix of both? Which software and/or hardware
% blocks are included in your design, and what hardware device (if any) did you target? In most cases, it
% would be helpful to include block diagrams of your implementation illustrating the flow of data through
% your design, the interconnection between different blocks, and whether each block is implemented
% in software or hardware. As in the previous section, providing meaningful visualizations would help
% the reader better appreciate your work. Please also include one or two interesting aspects of your
% implementation, especially any specific implementation strategies necessary for creating a functionally
% correct design with good performance.
\subsubsection{SHA-512}

Before we created the password cracker, we implemented the SHA-512 core on its own, first to ensure
correctness, and then as a way to optimize it in isolation. We implemented SHA-512 in software first,
using the algorithm presented in the "Descriptions of SHA-256, SHA-384, and SHA-512" paper that is linked
in the description of \ref{SHA512}. One of the key aspects of SHA-512 is that the message bytes need to
be in big-endian order when they are processed by the compression function. In order to accomplish this
in the software version, we used the \verb htobe64() function provided by Linux. This function will not
synthesize, however, since it relies on architecture specific assembly instructions. We realized that
this was not an issue since we could control exactly how the bytes of the message were sent/received from
the host to the FPGA. The rest of the software implementation was synthesizeable. We decided to only
implement the algorithm for hashing one message block on the FPGA, and put all of the pre-processing
including splitting the message up into blocks on the host. The host the sent intermediate hash and a
message block to the FPGA, which then performed the computation described in \ref{SHA512} and sent back
the result hash.

We made a number of optimizations to the baseline SHA-512 algorithm that we started with. One of the
most important ones recognizing that the $i^{th}$ block of $W$ only relies on the 15 preceeding blocks.
Since each block of $W$ is accessed sequentially, we can use a 16-block block array as a shfit register
to store the values we need, instead of the entire 80 block array. Since the first 16 values in $W$ are
just the current message chunk with no additional processing, we can simultaneously do the first 16
rounds of the compression function and also fill $W$ in parallel, and then do the remaining $64$ rounds
computing one new $W$ value each time.

\subsubsection{Unix Password Hashing}

\subsection{Evaluation}
% Students should describe the experimental setup used to evaluate their design. Students
% should describe the data inputs used to evaluate their design and provide an analysis of the achieved
% results. The results should be clearly summarized in terms of tables, text, and/or plots. Please provide
% qualitative and quantitative analysis of the results and discuss insights from these results. Results may
% include (but are not limited to) the execution time of an algorithm, hardware resource usage, achievable
% throughput, and error rate. It would be interesting, for example, to discuss why one design is better
% than another, why one design achieves a higher metric than another, or how you trade-off one metric
% for another. Consider going into detail for one particular instance of your experiment and analyze how
% it achieves the given results.
\begin{table}[h]
\begin{tabular}{@{}llllll@{}}
\toprule
Version                   & Latency (cycles) & BRAM Usage & DSP Usage & FF Usage & LUT Usage \\ \midrule
baseline                  & 534              & 2\%         & 0\%        & 0\%       & 5\%    \\  
baseline-pipeline-1       & 299              & 2\%         & 0\%        & 1\%       & 6\%    \\ 
baseline-pipeline-2       & 235              & $<$1\%      & 0\%        & 9\%       & 20\%   \\
shift-register            & 2373             & 2\%         & 0\%        & 1\%       & 9\%    \\
shift-register-pipeline-1 & 91               & $<$1\%      & 0\%        & 3\%       & 12\%   \\
final-opt                 & 88               & $<$1\%      & 0\%        & 4\%       & 7\%   \\ \bottomrule
\end{tabular}
\caption{Usage and cycle counts for different version of our SHA-512 implementation. The baseline
version are nealy indentical to the software implementations, but they make use of optimization
directives. The shift-register versions use the optimization described in the Techniques section to
reduce the size of W and to remove the extraneous loops, in addition to making use of optimization
directives}
\label{table:shaversions}
\end{table}

